\documentclass{article}
\usepackage[left=1in, right=1in, top=1in, bottom=1in]{geometry}
\usepackage{graphicx}
\usepackage{subfig}
\usepackage{amsmath}

\title{One-trait, one-rate Ornstein-Uhlenbeck (with three optima) calibrated validation}
\author{F\'{a}bio K. Mendes}
\date{May 2019}

\begin{document}
\maketitle

\newpage

\section{Multivariate-normal likelihood class}

Priors for simulations were $\sigma^2 \sim \text{Exp}(5)$, $y_0 \sim
\text{N}(0, 2)$ (root value), $\theta_1, \theta_2, \theta_3 \sim \text{N}(1, 2)$, and $\alpha \sim
\text{LN}(1, 1)$.
We simulated a single tree (50 tips) from a Yule prior and fixed it.

\begin{figure}[h]
  \centering
  \includegraphics[width=10cm]{../OUMVNThreeOpt_ultra_graphs.png}
\end{figure}

\begin{figure}[h]
  \centering
  \includegraphics[width=5cm]{../OUMVNThreeOpt_ultra_tree.png}
\end{figure}

\begin{center}
\begin{tabular}{c | c}
  Parameter & Coverage (\%) \\\hline
  $\sigma^2$ & 95\\
  $y_0$ & 97\\
  $\theta_1$ & 94\\
  $\theta_2$ & 98\\
  $\theta_3$ & 87\\
  $\alpha$ & 92
\end{tabular}
\end{center}

\clearpage
\noindent We re-simulated data on this other tree with approximately half of the tips made short manually (all branch lengths = 0.1).

\begin{figure}[h]
  \centering
  \includegraphics[width=10cm]{../OUMVNThreeOpt_nonultra_graphs.png}
\end{figure}

\begin{figure}[h]
  \centering
  \includegraphics[width=5cm]{../OUMVNThreeOpt_nonultra_tree.png}
\end{figure}

\begin{center}
\begin{tabular}{c | c}
  Parameter & Coverage (\%) \\\hline
  $\sigma^2$ & 91\\
  $y_0$ & 92\\
  $\theta_1$ & 94\\
  $\theta_2$ & 96\\
  $\theta_3$ & 97\\
  $\alpha$ & 93
\end{tabular}
\end{center}
\end{document}
